The \textbf{advection equation (AE)} is a simple hyperbolic partial differential equation, which in 1D is given by:
\begin{equation}
    \frac{\partial{u}}{\partial{t}} + a \frac{\partial{u}}{\partial{x}} = 0 \:.
    \label{eq:advection}
\end{equation}
Its analytical solution is $u(x, t) = u(x-at, 0)$, and thus $a$ represents the velocity at which the function $u(x, t)$ shifts. We will consider $a={\rm const}=1$, $x \in \left[0,10\right]$ and $t \in \left[0,20\right]$.

\vspace{4mm}  

The first function we want to study is a gaussian with initial location $x_0=5$ and scale $\sigma=1/\sqrt{2}$:
\begin{equation}
    u(x, t=0) = \exp{\left[-(x-x_0)^2\right]} \:.
    \label{eq:gaussian}
\end{equation}
We would like to find a numerical method which preserves both the shape and the amplitude of the solution in time. We can plot, for example, the \textbf{L2-norm}:
\begin{equation} 
    \norm{u_j^n}_2 = \left(\frac{1}{J} \sum_{i = 1}^J \abs{u_i^n}^2\right)^{1/2} \:,
    \label{eq:l2norm}
\end{equation}
whose evolution over time should ideally be a horizontal line.




\subsection{FTCS} \label{advection_FTCS}



